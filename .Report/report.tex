\documentclass{article}
\usepackage[hidelinks]{hyperref} %for hyperlinks
\usepackage{graphicx} %for images
\usepackage[backend=biber, style=apa]{biblatex} %bibliography
\usepackage{longtable} %allows comically long tables
\usepackage{changepage}
\addbibresource{references.bib}

\hypersetup{
    colorlinks   = true, %Colours links instead of ugly boxes
    urlcolor     = blue, %Colour for external hyperlinks
    linkcolor    = blue, %Colour of internal links
    citecolor   = red %Colour of citations
  }

\title{Web Application Project}
\author{Richard Hughes}
\date{October 9, 2024 - \today}
% \setlength{\parindent}{0pt}
% \setlength{\parskip}{16pt}

\begin{document}
\maketitle
\newpage

%TODO: Add the admin page logins
\section*{Preface}
  The development history spans across two GitHub repositories: 
  \href{https://github.com/richardh05/Web-Application-Development-J110823}{my first attempt} and 
  \href{https://github.com/richardh05/Carlos-Pizza}{my final attempt}.
  The reasoning for this will be detailed in the report, and I will attempt to merge the commit history once I am more experienced with Git. The database can be accessed with the credentials 
  \verb|user_db_2326227_carlosdb| \& \verb|ID398jQIPoBokrfW|. The website itself can be controlled using the admin credentials
  \verb|admin@ucm.ac.im| \& \verb|P@55word|, along with a customer account:
  \verb|example@example.com| \& \verb|Password123!|. 
  The project can be viewed as a \href{https://2326227.win.studentwebserver.co.uk/CO5227}{live webpage} via Plesk. Though uncited in the project code, the menu item images are taken from  \textcite{DominosMenu}.

\tableofcontents
\newpage


\section{Testing}
  \subsection{My project: 28.02.25}
  \begin{adjustwidth}{-2cm}{-2cm}
  \begin{center} %[\textwidth]{%
    \begin{longtable}{|p{1.2cm}|c|p{5cm}|p{5cm}|c|}
      \hline Test & Grade & Reasons & Recommendations \\ \hline 
      \endfirsthead
      
      \hline Test & Pass & Reasons & Recommendations \\ \hline 
      \endhead

      Layout & Pass & 
      The layout is used in every page, with intuitive navigation. Admin sections can be navigated to if the user has sufficient access. &
      Try to move the ``Checkout'' button to the right side, so that it's aligned with the other account controls. \\ \hline

      Style & Pass & 
      The \texttt{site.css} file is used in every page, styling the majority of elements the user can see. &
      Use other colours in the design language, particularly green, to differentiate ``Confirm'' and ``Discard'' buttons. \\ \hline
      
      Item Index & Pass &
      The item index displays flex-boxes with images and details, which are all searchable. &
      Group items by their category using headings, for improved navigation. \\ \hline

      Item Details & Pass &
      The details of menu items can be viewed after clicking the flex-box, and are formatted neatly. &
      Implement an ``Add to basket'' button within the ``details'' page for improved navigation.\\ \hline

      Edit Items & Pass &
      The menu items can be edited and saved, with checkboxes for boolean values as extra validation. Images can be uploaded. &
      Add extra validation for the ``Category'' field, which only has three valid options despite using a textbox. \\ \hline

      Create Items & Pass &
      The creation menu matches the editing menu. &
      Creation could also benefit from extra validation. \\ \hline

      Delete Items & Pass & 
      The user is asked to confirm the deletion, warning them before starting the process. &
      Use a shared ``partial view'' for the details and deletion menu, to make the code more DRY.\\ \hline

      Contact & Pass &
      An interactive JavaScript map and validated contact form is displayed. If JavaScript is disabled, a static map is displayed. &
      Restructure the page with some text, as currently it looks quite barren. \\ \hline

      Register & Pass &
      Accounts can be registered with validation and stored in the database. &
      Nothing to add. \\ \hline

      Login & Pass &
      Accounts can be logged into with validation and read from the database. The button is replaced with account settings once logged in. &
      Perhaps create a basket upon login if one does not exist, to resolve the issue of only new accounts being able to use checkout. \\ \hline

      Admin & Pass &
      A default admin account exists that can access an exclusive admin page. Editing, deleting, and creating items are restricted to this role. Admin functions are hidden to normal users. &
      Allow a way to easily create or appoint new admin users. \\ \hline

      Basket & Pass &
      A checkout page can be used to purchase items by newly created accounts. They must first be added to the basket from the Menu section, before the items are displayed in the checkout page. &
      One of the more critical bugs is that the checkout menu throws an error if accessed by an account created before the feature's implementation. This should be resolved. \\ \hline


    \end{longtable}
  \end{center}
\end{adjustwidth}

\subsection{Karst Café: 29.02.25}
Karst Café can be found at: \url{https://2326780.win.studentwebserver.co.uk/KarstCafe}
  \begin{adjustwidth}{-2cm}{-2cm}
  \begin{center} %[\textwidth]{%
    \begin{longtable}{|p{1.2cm}|c|p{5cm}|p{5cm}|c|}
      \hline Test & Grade & Reasons & Recommendations \\ \hline 
      \endfirsthead
      
      \hline Test & Pass & Reasons & Recommendations \\ \hline 
      \endhead

      Layout & Pass & 
      Layout page is quite close to the default, but still responsive. Seems to be used on all pages. &
      Use the Café's logo in place of the ``Karst Café'' text on the left. \\ \hline 

      Style & Pass & 
      The graphics (that seem to be AI generated) are stylish, and compliment each other well. &
      Try to use more CSS next time to compliment the aesthetic. \\ \hline
     
      Item Index & Pass &
      The content and search bar function perfectly. &
      The layout of the page does not hold up quite as much at different resolutions. Consider tweaking the CSS. \\ \hline

      Item Details & Pass &
      Fairly standard details page. I particularly like the vegetarian icon being inline with the text. &
      The ``Back to list'' link is off centre, being the only element on the left of the page. \\ \hline
      
      Contact & Pass &
      The map is in a separate page, which connects to Google successfully. The contact form has sound validation, particularly on the email address. &
      The pages are fairly lacking in content: add "Opening times", social media links, etc. \\ \hline
      
      Register & Pass &
      Registering a profile works as expected, with sound validation. &
      Storing the forename and surname may lead to unnecessary future complexity, although this is a matter of preference. \\ \hline
      
      Login & Pass &
      Login page works as expected, the option to reset password is particularly great. &
      Nothing to suggest here. \\ \hline
           
      Basket & Pass &
      Fantastic, the ability to change quantities and remove items from the basket page demonstrates self-study. &
      Try to show the individual cost of an item alongside the total cost of it's quantities, e.g. $2 \times \pounds12.99$ for a margherita pizza. \\ \hline

    \end{longtable}
  \end{center}
\end{adjustwidth}

\subsection{TheCoffeeBean: 29.02.25}
TheCoffeeBean can be found at: \url{https://2317350.win.studentwebserver.co.uk/TheCoffeeBean}
  \begin{adjustwidth}{-2cm}{-2cm}
  \begin{center} %[\textwidth]{%
    \begin{longtable}{|p{1.2cm}|c|p{5cm}|p{5cm}|c|}
      \hline Test & Grade & Reasons & Recommendations \\ \hline 
      \endfirsthead
      
      \hline Test & Pass & Reasons & Recommendations \\ \hline 
      \endhead

      Layout & Pass & 
      The layout page is great: a lot of thought has clearly gone into making it as aesthetically pleasing as possible &
      There seems to be an accidental HTML tag in the navigation bar. \\ \hline 

      Style & Pass & 
      Very pleasing style, I now wish I had implemented transparency in my own project. &
      It seems images are rendered at their full size rather than being assigned a max width, not sure if this is intentional or not. \\ \hline
     
      Item Index & Pass &
      The flex boxes have controls for quantity, and display the item information adequately &
      The item images do not seem to load correctly on the web server. \\ \hline

      Item Details & Fail &
      I could not find any way to access a details view for the items. &
      Try re-scaffolding the details page, if not done previously. \\ \hline
      
      Contact & Fail &
      The about page link doesn't work from the navigation bar. Inputting the address manually does not seem to work either. &
      Add a basic contact page with some opening times and an HTML form. Don't worry about JavaScript if it is too ambitious. \\ \hline
      
      Register & Fail &
      Trying to register throws an error, requesting swapping to a development environment. &
      Storing the forename and surname may lead to unnecessary future complexity, although this is a matter of preference. \\ \hline
      
      Login & Pass &
      Logging in from the created account works in spite of the error message. &
      Nothing to suggest here. \\ \hline
           
      Basket & Fail &
      I couldn't seem to see my added items in the basket, suggesting the logic is not implemented. &
      Follow the tutorials on moodle. \\ \hline

    \end{longtable}
  \end{center}
\end{adjustwidth}

\section{Development Log}
\subsection{Homepage \& styling: 06.10.2024}
I have created the project, chosen the theme, developed the navigation bar, created all page files, created the logo and styled the template page. More or less everything that was attempted, besides some nitpicks with the styling, is working well. Things that could be improved include the favicon, changing text in the navigation bar and positioning of the footer. I am currently using workarounds for these.

\subsection{Displaying items: 07.12.2024}
 Due to an error that Jules and I are unable to solve, and have been unable to for weeks, it is likely that I will need to restart my project. This is of course far from ideal, however the code comments, GitHub history and sections of the report I have completed so far should allow me to recreate my progress step-by-step. I will have to put in some overtime to make this happen, of course, but it seems that’s the only path forwards. However, Plesk is generally working well. The FTP connection works throughout the whole IDE, which will be helpful when starting the second attempt.


\subsection{Adding, deleting \& editing items: 20.01.2025}
Due to continued personal issues and other assignments requiring my attention, I have been unable to continue with the web application. These issues will be clearing up soon with the four assignments of January ending on the 23rd, at which point I will be able to place all my focus into the programming-heavy projects at the tail end of the year.

\subsection{Registration \& logins: 19.02.2025}
After committing several weeks to the project, I am pleased to report that I am nearly fully caught up to where I should be at this point. After restarting the project, I was able to move in my HTML and CSS from the first attempt, connect the database and Plesk server, created my context and models, added menu items displaying and searching, and started work on registration and login. Accounts now work correctly, although roles need to be implemented. Before writing the report, I encountered a bug where the footer would overwrite some main content of the page. I also need to implement roles and purchasing. The contact page must also be completed, with a JavaScript map to show the location of the restaurant.

\section{Reflection}
\subsection{Added content}
Since the final milestone I have managed to fully implement the Identity framework and roles, specifically ``Admin'' and ``User Manager''. Another major feature implemented is purchasing: the database, UI/UX, and backend code have all been reworked to accommodate. Image uploading is available by adding or editing a menu item, and can be viewed from the menu or home page. Items designated as unavailable can also only be seen by admins, in the updated admin page. Admin and CRUD pages are also now restricted to Admin users. I am pleasantly surprised I managed to get it all done, as I have not had the best of luck with the project overall.

\subsection{Thoughts}
Building the web application for Carlos Pizza has been a rewarding yet challenging experience. While I have enjoyed tackling the complexities that come with working in the framework, there have been moments of frustration, especially when debugging and resolving technical issues. The learning curve was steep and overwhelming, however overcoming these hurdles has been fulfilling, and I have gained a deeper understanding of web development. Given more time and without the pressure of an impending deadline, I feel I could come to quite enjoy ASP.NET Core. Despite the challenges, I have found a sense of accomplishment in pushing through and seeing the project come together. That said, I am intrigued by the possibility of trying Django in the future, as I have heard it offers a simpler, more streamlined approach to web development. I have been using Python for hobby projects recently, and am excited about the idea of working with a framework that allows me to focus on building interesting features rather than fighting with the underlying system. If I ever revisit ASP.NET, I'll likely try a simpler setup with an SQLite database and Visual Studio on Windows, reducing the number of variables that can go wrong.

\subsection{Conclusion}
Despite facing several setbacks throughout the course of this project, including technical difficulties, personal challenges, and the need to restart the work, significant progress has been made. The foundational elements of the web application, such as the Identity integration, database setup, and admin controls are now in place. While there are still tasks to complete, many of which I have listed in the repository's ``Issues'' page, the project's trajectory is now much clearer. With a solid foundation established, and key issues resolved, I am confident that I can complete the remaining features within the projected timeline. I was hoping I would have made more progress towards some smaller criteria at this point, such as being able to remove items from basket, however the experience has taught me an important lesson in time management Going forward, the focus will be on polishing the user experience, refining functionality, and ensuring that all remaining components are integrated seamlessly.

\nocite{*}
\printbibliography[heading=bibnumbered]
\end{document}